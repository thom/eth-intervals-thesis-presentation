%==============================================================================
% presentation.tex
%==============================================================================


%==============================================================================
% Configuration
%==============================================================================

% Internationalisation
\usepackage[utf8]{inputenc}
\usepackage[T1]{fontenc}
% \usepackage[ngerman]{babel}

% Miscellaneous packages
\usepackage{url}
\usepackage{color,listings,paralist}
\usepackage{enumerate}
\usepackage{tabularx}
\usepackage{alltt}
\usepackage{wasysym}

% Use default Acrobat reader fonts
\usepackage{mathpazo}

% Use CM fonts (increases document size)
\usepackage{ae}

% Use images
\usepackage{graphicx}

% Configure beamer
\usetheme[secheader]{Ikhono}
\usefonttheme[onlylarge]{structurebold}
\setbeamertemplate{navigation symbols}{}

% Variables
\providecommand{\Title}{An Advanced Scheduler for Intervals}
\providecommand{\Subtitle}{Master's Thesis}
\providecommand{\Author}{Thomas Weibel <weibelt@ethz.ch>}
\providecommand{\Institute}{Laboratory for Software Technology, \\
  Swiss Federal Institute of Technology Z\"urich}
\providecommand{\Date}{September 7, 2010}

% PDF settings
\hypersetup{
  pdftitle={\Title, \Subtitle},
  pdfauthor={\Author},
  pdfsubject={\Institute},
  pdfkeywords={Master's Thesis, Thomas Weibel,
    Intervals, Parallel Programming}
}

% Titlepage
\title{\Title}
\subtitle{\Subtitle}
\author{\Author}
\institute{\Institute}
\date{\Date}

% Listings
\lstdefinestyle{Default}{
  language=Java,
  tabsize=2,
  mathescape=true,
  inputencoding=utf8,
  showstringspaces=false,
  fontadjust=true,
  basicstyle=\ttfamily,
  keywordstyle=\color{blue}\bfseries,
}
\lstset{style=Default}


%==============================================================================
% Document
%==============================================================================

\begin{document}


% Titlepage
\begin{frame}[plain]
  \titlepage
\end{frame}

\note{
}


\section*{Introduction}

\begin{frame}{Executive Summary}
  \begin{itemize}
  \item TODO
  \end{itemize}
\end{frame}

\note{
}

\begin{frame}{Abstract}
  The goal of this project is to improve the efficiency of the
  Intervals scheduler and it consists of two parts:

  \vspace{\stretch{1}}

  \begin{enumerate}
  \item Explore and profile different work-stealing queue
    implementations
  \item Locality-aware work-stealing
  \end{enumerate}
\end{frame}

\note{ 
  Intervals are a new, higher-level primitive for parallel
  programming allowing programmers to directly construct the program
  schedule. They are under active development at ETH Zürich as part of
  the PhD research of Nicholas D. Matsakis.

  The intervals implementation in Java uses a work-stealing scheduler
  where a worker running out of work tries to steal work from
  others. The scope of this thesis is to improve the performance of
  the intervals scheduler.

  We implement and analyze an advanced scheduler for intervals. It is
  designed for locality-aware scheduling using locality hints provided
  by the programmer. Instead of employing work-stealing workers, our
  scheduler groups workers into Work-Stealing Places. Each
  work-stealing place has a fixed number of workers and a local deque
  to maintain ready tasks. The workers of a place share its local
  deque from which they obtain work. When a worker finds that the pool
  of its place is empty, it tries to steal a task from the pool of a
  victim place chosen at random. Locality-aware intervals are added to
  their preferred place once they are ready for scheduling.

  Providing locality hints to intervals is optional and the
  performance of locality-ignorant programs executed with the new
  scheduler implementation is comparable to that of the original
  scheduler.  
}

\begin{frame}{Intervals}
  TODO
\end{frame}

\note{
  Intervals are a new, higher-level primitive for multi-threaded
  programming allowing programmers to directly construct the program
  schedule. They are under active development at ETH Zürich as part of
  the PhD research of Nicholas D. Matsakis.

  The intervals implementation in Java uses a work-stealing scheduler
  where a worker running out of work tries to steal work from
  others. The scope of this thesis is to improve the performance of
  the intervals scheduler.

  Traditional primitives for synchronizing multi-threaded programs,
  such as semaphores and barriers, are low-level and dangerous to
  use. They are prone to errors, especially deadlocks and race
  conditions, and require careful attention to implementation details
  to achieve good performance.

  Intervals are a higher-level alternative that make parallel
  programming more secure while maintaining the flexibility and
  efficiency of threads. When using intervals, programmers create
  lightweight tasks and order them using \emph{happens before}
  relations. Programmers need not specify when tasks should block or
  acquire a lock. Instead they define when a task should execute in
  relation to other tasks, and what locks it should hold during
  execution. It is the duty of the runtime system to make this
  schedule pass.

  The intervals API supports arbitrary \emph{happens before} relations
  making the model very flexible. Intervals can be used to emulate
  existing thread primitives, but they can also be used to create
  program schedules for which no standard primitives exist, for
  example peer-to-peer synchronization.

  Intervals can be extended to support both runtime and static checks
  for errors like data race protection and deadlocks. An error in one
  task prevents other, dependent tasks from executing.
}

\begin{frame}{Work-Stealing Scheduler}
  TODO
\end{frame}

\note{
  The implementation of intervals for Java makes use of a
  work-stealing scheduler similar to those found in Cilk, Java 7,
  Intel Threading Building Blocks, or Microsoft Task Parallel Library
  but extended to support locks and happens before edges.

  A work-stealing scheduler employs a fixed number of threads called
  workers. Each worker has a local double-ended queue, or deque, to
  maintain its own pool of ready tasks from which it obtains
  work. When a worker finds that its pool is empty, it steals a task
  from the pool of a victim worker chosen at random.

  To obtain work, a worker takes the ready task from the tail of its
  deque and executes it. If the task terminates, the worker goes back
  to the tail of its deque to take another task upon which it can
  work. When assigning a new task to a worker, the worker puts the
  newly ready task onto the tail of its deque. Thus, as long as a
  worker's deque is not empty, the worker manipulates its deque in a
  LIFO (stack-like) manner.

  When a worker tries to obtain work by taking a task from the tail of
  its deque and it finds that it is empty, the worker becomes a
  thief. It picks a victim worker at random and attempts to obtain
  work by removing the task at the head of the victim's deque. If the
  victim's deque is empty, then the thief picks another victim worker
  and tries again until it finds a victim whose deque it not empty. At
  which point the thief continues to work on the stolen task as
  described above. Since steals take place at the head of the victim's
  deque, stealing operates in a FIFO manner.

  Accessing the run queues at different ends offers several advantages:

  \begin{itemize}
  \item It reduces contention by having owner and thieves working on
    opposite sides of the deque.
  \item Recursive divide-and-conquer algorithms generate ``large''
    tasks early. Thus, the older stolen task is likely to further
    provide more work to the stealing worker.
  \item Stealing a task also migrates its future workload, which helps
    to increase locality.
  \end{itemize}

  The assignment of tasks to workers for execution is done in a
  provably efficient manner.
}


\section{Locality-Aware Intervals Scheduling}

\begin{frame}{Outline}
  \tableofcontents[current]
\end{frame}

\note{
}

\begin{frame}{TODO}
  TODO
\end{frame}

\note{
  The current implementation of the intervals library uses a
  locality-ignorant work-stealing scheduler to schedule ready-to-run
  tasks. In this thesis we introduce LASSI, a locality-aware scheduler
  for intervals. The correct acronym would be LASI but we chose LASSI
  instead as we really enjoy drinking refreshing masala lassi \smiley

  % Scheduling

  In chip multiprocessor systems it may be more efficient to schedule
  a task on one processor than another. As modern CMPs feature a
  heterogeneous memory hierarchy where access times depend on which
  processor an interval is running, locality-aware intervals can lead
  to improved performance:

  \begin{itemize}
  \item By scheduling data sharing intervals on the same processor
    they perform prefetching of shared regions for one another.
  \item Scheduling non-communicating intervals with high memory
    footprints on different processors helps to reduce cache
    contention and potential cache capacity problems.
  \end{itemize}

  When using locality-ignorant work-stealing we cannot fully exploit
  the heterogeneous memory hierarchy of CMPs for our benefit. Thus, we
  implement and analyze LASSI, a locality-aware scheduler for
  intervals. LASSI is designed for locality-aware scheduling using
  locality hints provided by the programmer. Instead of employing
  work-stealing workers, it groups workers into \emph{Work-Stealing
    Places}.

  Each work-stealing place has a fixed number of workers and a local
  deque to maintain ready tasks. The workers of a place share its
  local deque from which they obtain work. When a worker finds that
  the pool of its place is empty, it tries to steal a task from the
  pool of a victim place chosen at random. Locality-aware intervals
  are added to their preferred place once they are ready for
  scheduling.

  Providing locality hints to intervals is optional and the
  performance of locality-ignorant programs executed with the new
  scheduler implementation is comparable to that of the original
  scheduler.

  We study the performance of LASSI with benchmarks using data sharing
  intervals. Our experimental results show that \emph{best locality}
  placement of intervals can achieve up to $1.15\times$ speedup over
  \emph{worst} or \emph{ignorant locality} placement. Cache hits can
  be increased by up to $1.5\times$ and cache misses can be reduced by
  up to $3.1\times$ for the benchmarks and platform studied in this
  thesis.
}


\section{Approach}

\begin{frame}{Outline}
  \tableofcontents[current]
\end{frame}

\note{
}

\begin{frame}{TODO}
  TODO
\end{frame}

\note{
}


\section{Implementation}

\begin{frame}{Outline}
  \tableofcontents[current]
\end{frame}

\note{
}

\begin{frame}{TODO}
  TODO
\end{frame}

\note{
}


\section{Performance Evaluation}

\begin{frame}{Outline}
  \tableofcontents[current]
\end{frame}

\note{
}

\begin{frame}{TODO}
  TODO
\end{frame}

\note{
}


\section{Related Work}

\begin{frame}{Outline}
  \tableofcontents[current]
\end{frame}

\note{
}

\begin{frame}{TODO}
  TODO
\end{frame}

\note{
}


\section{Conclusions and Future Work}

\begin{frame}{Outline}
  \tableofcontents[current]
\end{frame}

\note{
}

\begin{frame}{TODO}
  TODO
\end{frame}

\note{
}


\section*{Outro}

\begin{frame}{Summary}
  \begin{itemize}
  \item TODO
  \end{itemize}
\end{frame}

\note{
}

\begin{frame}
  \begin{center}
    \huge Questions?
  \end{center}
\end{frame}

\note{
}

\end{document}
