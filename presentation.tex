%==============================================================================
% presentation.tex
%==============================================================================


%==============================================================================
% Configuration
%==============================================================================

% Internationalisation
\usepackage[utf8]{inputenc}
\usepackage[T1]{fontenc}
% \usepackage[ngerman]{babel}

% Different packages
\usepackage{url}
\usepackage{color,listings,paralist}
\usepackage{enumerate}
\usepackage{tabularx}
\usepackage{alltt}

% Use default Acrobat reader fonts
\usepackage{mathpazo}

% Use CM fonts (increases document size)
\usepackage{ae}

% Use images
\usepackage{graphicx}

% Configure beamer
\usetheme[secheader]{Ikhono}
\usefonttheme[onlylarge]{structurebold}
\setbeamertemplate{navigation symbols}{}

% Variables
\providecommand{\Title}{An Advanced Scheduler for Intervals}
\providecommand{\Subtitle}{Master's Thesis}
\providecommand{\Author}{Thomas Weibel <weibelt@ethz.ch>}
\providecommand{\Institute}{Laboratory for Software Technology, \\
  Swiss Federal Institute of Technology Z\"urich}
\providecommand{\Date}{September 7, 2010}

% PDF settings
\hypersetup{
  pdftitle={\Title, \Subtitle},
  pdfauthor={\Author},
  pdfsubject={\Institute},
  pdfkeywords={Master's Thesis, Thomas Weibel,
    Intervals, Parallel Programming}
}

% Titlepage
\title{\Title}
\subtitle{\Subtitle}
\author{\Author}
\institute{\Institute}
\date{\Date}

% Listings
\lstdefinestyle{Default}{
  language=Java,
  tabsize=2,
  mathescape=true,
  inputencoding=utf8,
  showstringspaces=false,
  fontadjust=true,
  basicstyle=\ttfamily,
  keywordstyle=\color{blue}\bfseries,
}
\lstset{style=Default}


%==============================================================================
% Document
%==============================================================================

\begin{document}


% Titlepage
\begin{frame}[plain]
  \titlepage
\end{frame}

\note{
}


\section*{Introduction}

\begin{frame}{Executive Summary}
  \begin{itemize}
  \item TODO
  \end{itemize}
\end{frame}

\note{
}

\begin{frame}{Abstract}
  The goal of this project is to improve the efficiency of the
  Intervals scheduler and it consists of two parts:

  \vspace{\stretch{1}}

  \begin{enumerate}
  \item Explore and profile different work-stealing queue
    implementations
  \item Locality-aware work-stealing
  \end{enumerate}
\end{frame}

\note{ 
  Intervals are a new, higher-level primitive for parallel
  programming allowing programmers to directly construct the program
  schedule. They are under active development at ETH Zürich as part of
  the PhD research of Nicholas D. Matsakis.

  The intervals implementation in Java uses a work-stealing scheduler
  where a worker running out of work tries to steal work from
  others. The scope of this thesis is to improve the performance of
  the intervals scheduler.

  We implement and analyze an advanced scheduler for intervals. It is
  designed for locality-aware scheduling using locality hints provided
  by the programmer. Instead of employing work-stealing workers, our
  scheduler groups workers into Work-Stealing Places. Each
  work-stealing place has a fixed number of workers and a local deque
  to maintain ready tasks. The workers of a place share its local
  deque from which they obtain work. When a worker finds that the pool
  of its place is empty, it tries to steal a task from the pool of a
  victim place chosen at random. Locality-aware intervals are added to
  their preferred place once they are ready for scheduling.

  Providing locality hints to intervals is optional and the
  performance of locality-ignorant programs executed with the new
  scheduler implementation is comparable to that of the original
  scheduler.  
}

\begin{frame}{Intervals}
  TODO
\end{frame}

\note{
}

\begin{frame}{Work-Stealing Scheduler}
  TODO
\end{frame}

\note{
}


\section{Locality-Aware Intervals Scheduling}

\begin{frame}{Outline}
  \tableofcontents[current]
\end{frame}

\note{
}

\begin{frame}{TODO}
  TODO
\end{frame}

\note{
}


\section{Approach}

\begin{frame}{Outline}
  \tableofcontents[current]
\end{frame}

\note{
}

\begin{frame}{TODO}
  TODO
\end{frame}

\note{
}


\section{Implementation}

\begin{frame}{Outline}
  \tableofcontents[current]
\end{frame}

\note{
}

\begin{frame}{TODO}
  TODO
\end{frame}

\note{
}


\section{Performance Evaluation}

\begin{frame}{Outline}
  \tableofcontents[current]
\end{frame}

\note{
}

\begin{frame}{TODO}
  TODO
\end{frame}

\note{
}


\section{Related Work}

\begin{frame}{Outline}
  \tableofcontents[current]
\end{frame}

\note{
}

\begin{frame}{TODO}
  TODO
\end{frame}

\note{
}


\section{Conclusions and Future Work}

\begin{frame}{Outline}
  \tableofcontents[current]
\end{frame}

\note{
}

\begin{frame}{TODO}
  TODO
\end{frame}

\note{
}


\section*{Outro}

\begin{frame}{Summary}
  \begin{itemize}
  \item TODO
  \end{itemize}
\end{frame}

\note{
}

\begin{frame}
  \begin{center}
    \huge Questions?
  \end{center}
\end{frame}

\note{
}

\end{document}
