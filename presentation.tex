%==============================================================================
% presentation.tex
%==============================================================================


%==============================================================================
% Configuration
%==============================================================================

% Internationalisation
\usepackage[utf8]{inputenc}
\usepackage[T1]{fontenc}
% \usepackage[ngerman]{babel}

% Different packages
\usepackage{url}
\usepackage{color,listings,paralist}
\usepackage{enumerate}
\usepackage{tabularx}
\usepackage{alltt}

% Use default Acrobat reader fonts
\usepackage{mathpazo}

% Use CM fonts (increases document size)
\usepackage{ae}

% Use images
\usepackage{graphicx}

% Configure beamer
\usetheme[secheader]{Ikhono}
\usefonttheme[onlylarge]{structurebold}
\setbeamertemplate{navigation symbols}{}

% Variables
\providecommand{\Title}{An Advanced Scheduler for Intervals}
\providecommand{\Subtitle}{Master's Thesis}
\providecommand{\Author}{Thomas Weibel <weibelt@ethz.ch>}
\providecommand{\Institute}{Laboratory for Software Technology, \\
  Swiss Federal Institute of Technology Z\"urich}
\providecommand{\Date}{\today}

% PDF settings
\hypersetup{
  pdftitle={\Title, \Subtitle},
  pdfauthor={\Author},
  pdfsubject={\Institute},
  pdfkeywords={
    Thomas Weibel, Intervals, Parallel Programming, Work-stealing Scheduler
  } 
}

% Titlepage
\title{\Title}
\subtitle{\Subtitle}
\author{\Author}
\institute{\Institute}
\date{\Date}

% Listings
\lstdefinestyle{Default}{
  language=Java,
  tabsize=2,
  mathescape=true,
  inputencoding=utf8,
  showstringspaces=false,
  fontadjust=true,
  basicstyle=\ttfamily,
  keywordstyle=\color{blue}\bfseries,
}
\lstset{style=Default}


%==============================================================================
% Document
%==============================================================================

\begin{document}


% Titlepage
\begin{frame}[plain]
  \titlepage
\end{frame}


\section*{Introduction}

% \begin{frame}{Executive Summary}
%   \begin{itemize}
%   \item TODO
%   \end{itemize}
% \end{frame}

\begin{frame}{Task Description}
  The goal of this project is to improve the efficiency of the
  Intervals scheduler and it consists of two parts:

  \vspace{\stretch{1}}

  \begin{enumerate}
  \item Explore and profile different work-stealing queue
    implementations
  \item Locality-aware work-stealing
  \end{enumerate}
\end{frame}


\section{Part 1: Work-stealing queue implementations}

\begin{frame}{Outline}
  \tableofcontents[current]
\end{frame}

\begin{frame}{Introduction}
  \begin{itemize}
  \item In a work-stealing scheduler, each worker keeps a pool of work
    items waiting to be executed
  \item The current Intervals implementation uses doubled-ended queues
  \item The first part of the thesis will be to explore alternative
    queue implementations and to compare them to the current
    implementation
  \end{itemize}
\end{frame}

\begin{frame}{\lstinline!LazyDeque!}
  \begin{itemize}
  \item TODO
  \end{itemize}
\end{frame}

\begin{frame}{\lstinline!WorkStealingLazyDeque!}
  \begin{itemize}
  \item TODO
  \end{itemize}
\end{frame}

\begin{frame}{\lstinline!WorkStealingDeque!}
  \begin{itemize}
  \item TODO
  \end{itemize}
\end{frame}

\begin{frame}{\lstinline!DuplicatingWorkStealingDeque!}
  \begin{itemize}
  \item TODO
  \end{itemize}
\end{frame}

\begin{frame}{\lstinline!IdempotentWorkStealingDeque!}
  \begin{itemize}
  \item TODO
  \end{itemize}
\end{frame}

\begin{frame}{\lstinline!DynamicWorkStealingDeque!}
  \begin{itemize}
  \item TODO
  \end{itemize}
\end{frame}

\begin{frame}{Results}
  \begin{itemize}
  \item TODO
  \end{itemize}
\end{frame}


\section{Part 2: Locality-aware work-stealing}

\begin{frame}{Outline}
  \tableofcontents[current]
\end{frame}

\begin{frame}{Introduction}
  \begin{itemize}
  \item Intervals scheduler randomly schedules work items
  \item When stealing a work item, the worker chooses his victim by
    ``random''
  \item To improve performance and reduce cache misses, work items
    that access the same data should be scheduled on the same worker
    or a worker ``nearby''
  \item In locality-aware work-stealing, each work item can be given
    an affinity for a worker, and when a worker obtains a work item it
    gives priority to the work items with affinity to it
  \end{itemize}
\end{frame}

\begin{frame}{TODO}
  \begin{itemize}
  \item TODO
  \end{itemize}
\end{frame}

% \section*{Outro}

% \begin{frame}{Summary}
%   \begin{itemize}
%   \item TODO
%   \end{itemize}
% \end{frame}

\end{document}
