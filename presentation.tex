%==============================================================================
% presentation.tex
%==============================================================================


%==============================================================================
% Configuration
%==============================================================================

% Internationalisation
\usepackage[utf8]{inputenc}
\usepackage[T1]{fontenc}
% \usepackage[ngerman]{babel}

% Different packages
\usepackage{url}
\usepackage{color,listings,paralist}
\usepackage{enumerate}
\usepackage{tabularx}
\usepackage{alltt}

% Use default Acrobat reader fonts
\usepackage{mathpazo}

% Use CM fonts (increases document size)
\usepackage{ae}

% Use images
\usepackage{graphicx}

% Configure beamer
\usetheme[secheader]{Ikhono}
\usefonttheme[onlylarge]{structurebold}
\setbeamertemplate{navigation symbols}{}

% Variables
\providecommand{\Title}{An Advanced Scheduler for Intervals}
\providecommand{\Subtitle}{Master's Thesis}
\providecommand{\Author}{Thomas Weibel <weibelt@ethz.ch>}
\providecommand{\Institute}{Laboratory for Software Technology, \\
  Swiss Federal Institute of Technology Z\"urich}
\providecommand{\Date}{\today}

% PDF settings
\hypersetup{
  pdftitle={\Title, \Subtitle},
  pdfauthor={\Author},
  pdfsubject={\Institute},
  pdfkeywords={
    Thomas Weibel, Intervals, Parallel Programming, Work-stealing Scheduler
  } 
}

% Titlepage
\title{\Title}
\subtitle{\Subtitle}
\author{\Author}
\institute{\Institute}
\date{\Date}

% Listings
\lstdefinestyle{Default}{
  language=Java,
  tabsize=2,
  mathescape=true,
  inputencoding=utf8,
  showstringspaces=false,
  fontadjust=true,
  basicstyle=\ttfamily,
  keywordstyle=\color{blue}\bfseries,
}
\lstset{style=Default}


%==============================================================================
% Document
%==============================================================================

\begin{document}


% Titlepage
\begin{frame}[plain]
  \titlepage
\end{frame}


\section*{Introduction}

% \begin{frame}{Executive Summary}
%   \begin{itemize}
%   \item TODO
%   \end{itemize}
% \end{frame}

\begin{frame}{Task Description}
  The goal of this project is to improve the efficiency of the
  Intervals scheduler and it consists of two parts:

  \vspace{\stretch{1}}

  \begin{enumerate}
  \item Explore and profile different work-stealing queue
    implementations
  \item Locality-aware work-stealing
  \end{enumerate}
\end{frame}


\section{Part 1: Work-stealing queue implementations}

\begin{frame}{Outline}
  \tableofcontents[current]
\end{frame}

\begin{frame}{Introduction}
  \begin{itemize}
  \item In a work-stealing scheduler, each worker keeps a pool of work
    items waiting to be executed
  \item The current Intervals implementation uses doubled-ended queues
  \item The first part of the thesis will be to explore alternative
    queue implementations and to compare them to the current
    implementation
  \end{itemize}
\end{frame}

\begin{frame}{\lstinline!LazyDeque!}
  \begin{itemize}
  \item Original implementation
  \item Uses \lstinline!AtomicReferenceArray! for tasks
  \item Unbounded and almost lock-free (\lstinline!steal()! and
    \lstinline!expand()! only allow one thief)
  \item Owner of the deque only lazily updated the location of the
    deque head, i.e. only when it tries to take something and finds it
    gone
  \end{itemize}
\end{frame}

\begin{frame}[fragile]{\lstinline!WorkStealingLazyDeque!}
  \begin{itemize}
  \item Based on \lstinline!LazyDeque!
  \item \lstinline!LazyDeque! uses
    \lstinline!tasks.getAndSet(index, null)! to get a queue element
  \item \lstinline!AtomicReferenceArray!:
\begin{lstlisting}[basicstyle=\fontsize{9}{11}\selectfont\ttfamily]
public final E getAndSet(int i, E newValue) {
  while (true) {
    E current = get(i);
    if (compareAndSet(i, current, newValue))
      return current;
  }
}
\end{lstlisting}
  \item Instead of using \lstinline!getAndSet!, this deque uses
    \lstinline!get! and \lstinline!compareAndSet!
  \end{itemize}
\end{frame}

\begin{frame}{\lstinline!WorkStealingDeque!}
  \begin{thebibliography}{10}
    \beamertemplatearticlebibitems
    
  \bibitem{dynamic}
    Chase, D.R., Lev, Yossi, {\em Dynamic circular work-stealing deque},
    \url{http://portal.acm.org/citation.cfm?id=1073974}
  \end{thebibliography}

  \vspace{\stretch{1}}

  \begin{itemize}
  \item Unbounded work-stealing deque implementation
  \item Uses cyclic non-atomic reference array for tasks
  \item \lstinline!top! is an atomic integer
  \item Lock-free, uses Compare-and-Swap
  \end{itemize}
\end{frame}

\begin{frame}{\lstinline!DuplicatingWorkStealingDeque!}
  \begin{thebibliography}{10}
    \beamertemplatearticlebibitems
    
  \bibitem{duplicating}
    Leijen, D, et al., {\em The design of a task parallel library},
    \url{http://portal.acm.org/citation.cfm?id=1640106&amp;dl=ACM}
  \end{thebibliography}

  \vspace{\stretch{1}}

  \begin{itemize}
  \item Uses cyclic non-atomic reference array for tasks
  \item Potentially returns a pushed element more than once:
    \begin{itemize}
    \item[$\rightarrow$] \lstinline!put! and \lstinline!steal! behave
      like normal, but \lstinline!take! is allowed to either take an
      element (and remove it from the queue), or just duplicate an
      element in the queue
    \end{itemize}
  \item Each task is associated with a state
    \begin{itemize}
    \item [$\rightarrow$] \lstinline!Init!, \lstinline!Running! or
      \lstinline!Done!
    \end{itemize}
  \item When executing a task, we first check whether it has been run
    already
  \item \lstinline!steal()! and \lstinline!expand()! synchronized,
    \lstinline!take()! is synchronized if there's only one item in the
    queue
  \item Variation: eager execution instead of queue expansion
  \end{itemize}
\end{frame}

\begin{frame}{\lstinline!IdempotentWorkStealingDeque!}
  \begin{thebibliography}{10}
    \beamertemplatearticlebibitems
    
  \bibitem{idempotent}
    Michael, M.M., et al., {\em Idempotent work stealing},
    \url{http://portal.acm.org/citation.cfm?id=1594835.1504186}
  \end{thebibliography}

  \vspace{\stretch{1}}  

  \begin{itemize}
  \item Same basic idea as \lstinline!DuplicatingWorkStealingDeque!
  \item Uses cyclic non-atomic reference array for tasks
  \item Lock-free
  \item Three implementations: \lstinline!FIFOQueue!,
    \lstinline!LIFOQueue!, \lstinline!Deque!
  \item \lstinline!Deque! performs the best
  \end{itemize}
\end{frame}

\begin{frame}{\lstinline!DynamicWorkStealingDeque!}
  \begin{thebibliography}{10}
    \beamertemplatearticlebibitems
    
  \bibitem{idempotent}
    Hendler, Danny, et al., {\em A dynamic-sized nonblocking work stealing deque},
    \url{http://www.springerlink.com/index/Y7HQ174L92170355.pdf}
  \end{thebibliography}

  \vspace{\stretch{1}}  

  \begin{itemize}
  \item Uses linked array nodes for tasks
  \item Lock-free implementation
  \item Usually slower than the other work-stealing queue implementations
    \begin{itemize}
    \item[$\rightarrow$] Implementation is quite complex
    \end{itemize}
  \end{itemize}
\end{frame}

\begin{frame}{Results}
  \begin{itemize}
  \item Biggest bottleneck: Local FIFO queue for pending work (try to
    get rid of it)
  \item Idempotent/duplicating queues are almost as fast as other
    queue implementations
    \begin{itemize}
    \item[$\rightarrow$] might use them in the second part
    \end{itemize}
  \item Original implementation was very good already
    \begin{itemize}
    \item[$\rightarrow$] small improvement by using \lstinline!get!
      and \lstinline!compareAndSet! instead of \lstinline!getAndSet!
    \end{itemize}
  \end{itemize}
\end{frame}

\begin{frame}{Results: Cache-misses}
  \begin{itemize}
  \item Work-stealing scheduler: More L1 cache misses than threads
    implementation
    \begin{itemize}
    \item[$\rightarrow$] but fewer L2 cache misses
    \end{itemize}
  \item Work-stealing scheduler: L1 and L2 cache miss ratio is very
    low
    \begin{itemize}
    \item[$\rightarrow$] besides an increase in cache misses, there's
      also an increase in cache references (array index check)
    \end{itemize}
  \end{itemize}  
\end{frame}


\section{Part 2: Locality-aware work-stealing}

\begin{frame}{Outline}
  \tableofcontents[current]
\end{frame}

\begin{frame}{Introduction}
  \begin{itemize}
  \item Intervals scheduler randomly schedules work items
  \item When stealing a work item, the worker chooses his victim by
    ``random''
  \item To improve performance and reduce cache misses, work items
    that access the same data should be scheduled on the same worker
    or a worker ``nearby''
  \item In locality-aware work-stealing, each work item can be given
    an affinity for a worker, and when a worker obtains a work item it
    gives priority to the work items with affinity to it
  \end{itemize}
\end{frame}

\begin{frame}{Literature Review}
  \begin{itemize}
  \item Blumofe et al.: Locality-guided work stealing (initial
    placement)
    \begin{itemize}
    \item[$\rightarrow$] each process maintains a mailbox: FIFO queue
      of pointers to threads that have affinity for the process
    \end{itemize}
  \item SLAW scheduler (Habanero Java): Same idea as Blumofe
  \item X10: Places to partition the global address space
  \item Mely: Locality-aware stealing heuristics
    \begin{itemize}
    \item[$\rightarrow$] aims at improving cache usage by minimizing
      the costs of cache misses
    \item[$\rightarrow$] orders cores by their ``distance'' from the
      stealing core
    \end{itemize}
  \end{itemize}
\end{frame}

% \section*{Outro}

% \begin{frame}{Summary}
%   \begin{itemize}
%   \item TODO
%   \end{itemize}
% \end{frame}

\end{document}
